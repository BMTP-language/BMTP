\documentclass[11pt]{ctexrep}
\usepackage[left=2.5cm,right=2.5cm,top=2.0cm,bottom=2.0cm]{geometry}
\usepackage[perpage]{footmisc}
\usepackage{xcolor}
\setCJKmainfont{Source Han Serif SC}
\newcounter{typelist}
\newcommand{\circnum}{\addtocounter{typelist}{1}\textcircled{\tiny\thetypelist}}
\begin{document}
\title{BMTP语言草案}
\author{张子辰}
\date{\today}
\maketitle
\tableofcontents
\chapter{数据类型草案}

\section{数据类型的归类}
\begin{flushright}
\sf
编号:100101-20190623
\end{flushright} 

数据类型的归类可以方便编译器知道数据类型的含义,从而使用恰当的方式完成多个数据类型的混合运输和隐性转化。

数据类型的分类有两级:类型之类型(typetype)和子类型之类型(subtypetype),每个数据类型还包含一个附加信息
——精度(accuracy)。

\subsection{类型之类型}

BMTP语言中包含以下基本的类型之类型:
\begin{enumerate}
\item number——数值
\item string——字符串
\item character——字符
\item container——容器
\item logic——与逻辑有关的值
\item function——函数
\item pointer——指针
\item refer——引用
\item empty——空
\end{enumerate}

\subsection{子类型之类型}

\begin{enumerate}
\item number
	\begin{enumerate}
	\item complex——复数
	\item float——浮点数
	\item fraction——分数
	\item integer——整数
	\item uinteger——无符号的整数
	\end{enumerate}
\item string
	\begin{enumerate}
	\item unicodestring——使用Unicode编码的字符串
	\item multibytestring——使用多字节编码\footnote{指区域性的编码。}的字符串
	\end{enumerate}
\item char
	\begin{enumerate}
	\item unicodechar
	\item multibytechar
	\end{enumerate}
\item container
	\begin{enumerate}
	\item array——数值
	\item vector——向量(每一个元素的类型之类型必须是number)
	\item matrix——矩阵(每一个元素的类型之类型必须是number)
	\item searchtree——搜索树
	\item heap——堆
	\item stack——栈
	\item queue——队列
	\item set——集合
	\item map——映射
	\end{enumerate}
\item logic
	\begin{enumerate}
	\item bool——布尔值
	\end{enumerate}
\item function
	\begin{enumerate}
	\item function
	\end{enumerate}
\item pointer
	\begin{enumerate}
	\item pointer
	\end{enumerate}
\item refer
	\begin{enumerate}
	\item refer
	\end{enumerate}
\item empty
	\begin{enumerate}
	\item empty
	\end{enumerate}
\end{enumerate}

\subsection{精度}

精度是恒量一个属于类型保持数据接近真实值的程度的量。精度用一个32位整数表示。一般地,数据的隐性类型转换向精度高的方向进行。特别地,精度为-1表示这个数据类型不适合被隐性转换为另一个类型。

精度这一语言特性在《运算符草案》还会提到。

\section{标准数据类型}
\subsection{数据类型列表}
\begin{enumerate}
\item number
	\setcounter{typelist}{0}
	\begin{enumerate}
	\item complex
		\begin{enumerate}
		\item[\circnum] complex32\hfill 30000
		\item[\circnum] complex64\hfill 40000
		\item[\circnum] complexmp\hfill 90000
		\end{enumerate}
	\item float
		\begin{enumerate}
		\item[\circnum] float32\hfill 30000
		\item[\circnum] float64\hfill 40000
		\item[\circnum] floatmp\hfill 90000
		\end{enumerate}
	\item fraction
		\begin{enumerate}
		\item[\circnum] frac32\hfill 30000
		\item[\circnum] frac64\hfill 40000
		\item[\circnum] fracmp\hfill 90000
		\end{enumerate}
	\item integer
		\begin{enumerate}
		\item[\circnum] int32\hfill 30000
		\item[\circnum] int64\hfill 40000
		\item[\circnum] intmp\hfill 90000
		\end{enumerate}
	\item uinteger
		\begin{enumerate}
		\item[\circnum] uint32\hfill 30000
		\item[\circnum] uint64\hfill 40000
		\item[\circnum] uintmp\hfill 90000
		\end{enumerate}
	\end{enumerate}
	\textbf{合计\thetypelist\ 个}
\item string
	\setcounter{typelist}{0}
	\begin{enumerate}
	\item unicodestring
		\begin{enumerate}
		\item[\circnum] stringutf8\hfill 10000
		\item[\circnum] stringutf16\hfill 20000
		\item[\circnum] stringutf32\hfill 30000
		\end{enumerate}
	\item multibytestring
		\begin{enumerate}
		\item[\circnum] stringmb\hfill -1
		\end{enumerate}
	\end{enumerate}
	\textbf{合计\thetypelist\ 个}
\item char
\setcounter{typelist}{0}
	\begin{enumerate}
	\item unicodechar
		\begin{enumerate}
		\item[\circnum] charutf8\hfill 10000
		\item[\circnum] charutf16\hfill 20000
		\item[\circnum] charutf32\hfill 30000
		\end{enumerate}
	\item multibytestring
		\begin{enumerate}
		\item[\circnum] charmb\hfill -1
		\end{enumerate}
	\end{enumerate}
	\textbf{合计\thetypelist\ 个}
\item container
\setcounter{typelist}{0}
	\begin{enumerate}
	\item array
		\begin{enumerate}
		\item[\circnum] array\hfill 10000
		\item[\circnum] deque\hfill 20000
		\end{enumerate}
	\item vector
		\begin{enumerate}
		\item[\circnum] vector\hfill -1 
		\end{enumerate}
	\item matrix
		\begin{enumerate}
		\item[\circnum] matrix\hfill -1 
		\end{enumerate}
	\item searchtree
		\begin{enumerate}
		\item[\circnum] bintree\hfill 10000
		\item[\circnum] treap\hfill 20000
		\item[\circnum] splay\hfill 30000
		\item[\circnum] avltree\hfill 40000
		\item[\circnum] rbtree\hfill 50000
		\end{enumerate}
	\item heap
		\begin{enumerate}
		\item[\circnum] binheap\hfill 30000
		\item[\circnum] leftistheap\hfill 40000
		\item[\circnum] pairheap\hfill 45000
		\item[\circnum] fibheap\hfill 50000
		\end{enumerate}
	\item stack
		\begin{enumerate}
		\item[\circnum] stack\hfill -1
		\end{enumerate}
	\item queue
		\begin{enumerate}
		\item[\circnum] queue\hfill -1
		\end{enumerate}
	\item set
		\begin{enumerate}
		\item[\circnum] treeset\hfill -1
		\item[\circnum] hashset\hfill -1
		\end{enumerate}
	\item map
		\begin{enumerate}
		\item[\circnum] treemap\hfill -1
		\item[\circnum] hashmap\hfill -1
		\end{enumerate}
	\end{enumerate}
	\textbf{合计\thetypelist\ 个}
\item logic
\setcounter{typelist}{0}
	\begin{enumerate}
	\item bool
		\begin{enumerate}
		\item[\circnum] bool\hfill -1
		\end{enumerate}
	\end{enumerate}
	\textbf{合计\thetypelist\ 个}
\item function
\setcounter{typelist}{0}
	\begin{enumerate}
	\item function
		\begin{enumerate}
		\item[\circnum] mathfun\hfill 10000
		\item[\circnum] function\hfill 20000
		\end{enumerate}
	\end{enumerate}
	\textbf{合计\thetypelist\ 个}
\item pointer
\setcounter{typelist}{0}
	\begin{enumerate}
	\item pointer
		\begin{enumerate}
		\item[\circnum] uniptr\hfill 20000
		\item[\circnum] tsptr\hfill 10000 
		\end{enumerate}
	\end{enumerate}
	\textbf{合计\thetypelist\ 个}
\item refer
\setcounter{typelist}{0}
	\begin{enumerate}
	\item refer
		\begin{enumerate}
		\item[\circnum] refer\hfill -1 
		\end{enumerate}
	\end{enumerate}
	\textbf{合计\thetypelist\ 个}
\item empty
\setcounter{typelist}{0}
\begin{enumerate}
	\item empty
		\begin{enumerate}
		\item[\circnum] void\hfill -1 
		\end{enumerate}
	\end{enumerate}
	\textbf{合计\thetypelist\ 个}
\end{enumerate}
\subsection{number}
\paragraph{complex}
\subparagraph{complex32}
32位复数,实部和虚部由两个float32的数据储存。
\subparagraph{complex64}
64位复数,实部和虚部由两个float64的数据储存。
\subparagraph{complexmp}
高精度复数,实部和虚部由两个floatmp的数据储存。
\paragraph{float}
\indent BMTP语言中的浮点数IEEE-754浮点类型。
\subparagraph{float32}
32为浮点数。
\end{document}