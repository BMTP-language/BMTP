\documentclass[11pt]{ctexrep}
\usepackage[left=2.5cm,right=2.5cm,top=2.0cm,bottom=2.0cm]{geometry}
\usepackage[perpage]{footmisc}
\setCJKmainfont{Source Han Serif SC}
\begin{document}
\title{BMTP语言草案}
\author{张子辰}
\date{\today}
\maketitle
\tableofcontents
\chapter{数据类型草案}

\section{数据类型的归类}
\begin{flushright}
\sf
编号:100101-20190623
\end{flushright} 

数据类型的归类可以方便编译器知道数据类型的含义,从而使用恰当的方式完成多个数据类型的混合运输和隐性转化。

数据类型的分类有两级:类型之类型(typetype)和子类型之类型(subtypetype),每个数据类型还包含一个附加信息
——精度(accuracy)。

\subsection{类型之类型}

BMTP语言中包含以下基本的类型之类型:
\begin{enumerate}
\item number——数值
\item string——字符串
\item character——字符
\item container——容器
\item logic——与逻辑有关的值
\item function——函数
\item pointer——指针
\item refer——引用
\item empty——空
\end{enumerate}

\subsection{子类型之类型}

\begin{enumerate}
\item number
	\begin{enumerate}
	\item complex——复数
	\item float——浮点数
	\item fraction——分数
	\item integer——整数
	\item uinteger——无符号的整数
	\end{enumerate}
\item string
	\begin{enumerate}
	\item unicodestring——使用Unicode编码的字符串
	\item multibytestring——使用多字节编码\footnote{指区域性的编码。}的字符串
	\end{enumerate}
\item char
	\begin{enumerate}
	\item unicodechar
	\item multibytechar
	\end{enumerate}
\item container
	\begin{enumerate}
	\item array——数值
	\item vector——向量(每一个元素的类型之类型必须是number)
	\item matrix——矩阵(每一个元素的类型之类型必须是number)
	\item searchtree——搜索树
	\item heap——堆
	\item stack——栈
	\item queue——队列
	\item set——集合
	\item map——映射
	\end{enumerate}
\item logic
	\begin{enumerate}
	\item bool——布尔值
	\end{enumerate}
\item function
	\begin{enumerate}
	\item function
	\end{enumerate}
\item pointer
	\begin{enumerate}
	\item pointer
	\end{enumerate}
\item refer
	\begin{enumerate}
	\item refer
	\end{enumerate}
\item empty
	\begin{enumerate}
	\item empty
	\end{enumerate}
\end{enumerate}

\subsection{精度}

精度是恒量一个属于类型保持数据接近真实值的程度的量。精度用一个32位整数表示。一般地,数据的隐性类型转换向精度高的方向进行。特别地,精度为-1表示这个数据类型不适合被隐性转换为另一个类型。

精度这一语言特性在《运算符草案》还会提到。
\end{document}